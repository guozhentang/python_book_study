\chapter{Numpy快速入门}
\section{先决条件}
在阅读本章之前,你需要对Python有一定的了解。如果你想回顾一下Python,可以参看\href{https://docs.python.org/}{Python 教程}。中文教程可以参考廖雪峰老师的教程:\href{https://www.liaoxuefeng.com/}{廖雪峰Python3 教程}。该教程通俗易懂,非常适合入门;教程最后的实战部分有一定难度,可以视情况而做。

如果你想运行本教程的示例,你必须预先安装一些必要的软件。详情请见\url{http://scipy.org/install.html}。也可以直接安装Python发行版\href{https://www.anaconda.com/download/}{anaconda}。

本书的运行环境是\textbf{Python 3.6.1 |Anaconda 4.4.0 (64-bit)}。

\section{基本知识}
Numpy中最主要的对象是ndarray,即$n$维数组。它封装了$n$维同类数据以及对这些数据的操作函数。他将相同的元素(一般来说都是一些数字)组合成一张表,用一个元祖来表示元素的索引。在Numpy中,维数被称为axes,即“轴”;轴的数量叫做rank,即“秩”。在线性代数中,一个矩阵A的列秩是A的线性独立的纵列的极大数。若A的秩为$n$,则A的列向量可以表示$n$维空间。

举个例子,3D坐标[1,2,1]是一个rank为1,axes长度是3的数组,他只有一维。接下来的例子中,rank为2(它是二维的)。第一维的长度是2,第二维的长度是3。
\begin{python}
[[ 1., 0., 0.],
[ 0., 1., 2.]]
\end{python}

Numpy的数组类叫做ndarray,通常也可以直接称作array。注意,numpy.array和Python标准的class类array.array是有区别的。array.array仅仅只能处理一维数组,且提供的功能函数少很多。ndarray中重要的属性有:
\begin{compactitem}
	\item \textbf{ndarray.ndim} 数组的维数,即秩
	\item \textbf{ndarray.shape} 数组的组成方式。以一个元祖的形式来表示每一个维度的大小。如果一个矩阵有$m$行$n$列,则shape可以表示成$(m,n)$。shape的元祖长度实际上就是ndim,本例中ndim=2
	\item \textbf{ndarray.size} 数组包含元素个数的总量。size值等于shape元祖里各个维度的乘积。size值并不表示数据占有的内存的大小,还需要知晓每个元素占有的内存大小
	\item \textbf{ndarray.dtype} dtype用于描述数组元素的类型。dtype既支持创建新的type,也支持Python标准的type。当然,Numpy也提供了自己的type,比如numpy.int32, numpy.int16, and numpy.float64
	\item \textbf{ndarray.itemsize}数组中每个元素的大小。比如,若数组中的元素的dtype是float64,则它的itemsize是8(=64/8),若数组中元素的dtype是comlex32,则它的itemsize是4(=32/8)。ndarray.itemsize与ndarray.dtype.itemsize值相等
	\item \textbf{ndarray.data} 数组中包含元素数据的buffer。一般来说我们没有必要直接用这个属性,因为我们一般是通过index来访问数据元素
\end{compactitem}

总结一下,一个ndarray类包含数据和处理数据的方法。数据中几个重要的属性是ndim、shape、size、dtype、itemsize、data。其中,shape可以推导出ndim。shape+dtype可以推导出除了data以外的其他全部属性。而data更接近底层访问数据的接口,可以用于和C语言等低级语言对接。
\subsection{一个例子}
\begin{python}
>>> import numpy as np
>>> a = np.arange(15).reshape(3, 5)
>>> a
array([[ 0,  1,  2,  3,  4],
       [ 5,  6,  7,  8,  9],
       [10, 11, 12, 13, 14]])
>>> a.shape
(3, 5)
>>> a.ndim
2
>>> a.dtype.name
'int32'
>>> a.itemsize
4
>>> a.size
15
>>> type(a)
<class 'numpy.ndarray'>
>>> b = np.array([6, 7, 8])
>>> b
array([6, 7, 8])
>>> type(b)
<class 'numpy.ndarray'>
>>> b.data
<memory at 0x0000000003494948>
\end{python}
\subsection{Numpy数组的创建}
有好几种创建Numpy数组的方法。首先,你可以用Numpy的array函数来创建一个Numpy数组,只需要给这个函数传入一个Python列表或者元祖作为参数。数组元素的dtype将从Python序列元祖的type中推导出来\footnote{若Python序列中有字符串和数字,经过推导后Numpy会统一成字符串类型}。
\begin{python}
>>> import numpy as np
>>> a = np.array([2,3,4])
>>> a
array([2, 3, 4])
>>> a.dtype
dtype('int32')
>>> b = np.array([1.2, 3.5, 5.1])
>>> b.dtype
dtype('float64')
\end{python}

array函数正确的用法应该是传入一个Python序列。一个经常犯的错误是直接给array函数出入多个数字作为参数,比如:
\begin{python}
>>> b = np.array([6, 7, 8])    #RIGHT
>>> b = np.array(6, 7, 8)      #WRONG
Traceback (most recent call last):
  File "<stdin>", line 1, in <module>
ValueError: only 2 non-keyword arguments accepted
\end{python}

若传入的参数是序列的序列,则array会将其转换为二维数组,以此类推,序列的序列的序列将被转换为三维数组\ldots
\begin{python}
>>> b = np.array([(1.5,2,3), (4,5,6)])
>>> c = np.array([((1,2),(3,4)), ((5,6),(7,8))])
>>> b
array([[ 1.5,  2. ,  3. ],
       [ 4. ,  5. ,  6. ]])
>>> c
array([[[1, 2],
        [3, 4]],

       [[5, 6],
        [7, 8]]])
\end{python}

数组元素的dtype也可以在创建数组的时候显式地指出:
\begin{python}
>>> d = np.array( [ [1,2], [3,4] ], dtype=complex )
>>> d
array([[ 1.+0.j,  2.+0.j],
       [ 3.+0.j,  4.+0.j]])
\end{python}

通常情况下,数组的元素的值刚开始是不确定的,但数组的大小却能定下来。因此,Numpy提供了一些可以创建固定大小数组的函数。数组元素的值可以用特定值填充占位或者直接使用随机值。这样的做法可以减少一些对于增长数组的需求,毕竟维护一个可增长数组是一个不小的开销。

numpy.zero和numpy.ones分别创建一个元素值全为0和1的数组,numpy.empty创建一个元素值随机的数组,元素的值根据当时的内存状态而视,不一定为0。采用这些方法创建数组的时候都可以指定dtype。若不指定dtype,则dtype默认是float64。
\begin{python}
>>> a=np.zeros( (3,4) )
>>> a
array([[ 0.,  0.,  0.,  0.],
       [ 0.,  0.,  0.,  0.],
       [ 0.,  0.,  0.,  0.]])
>>> a.dtype
dtype('float64')
>>> b=np.ones( (2,3,4), dtype=np.int16 )
>>> b
array([[[1, 1, 1, 1],
        [1, 1, 1, 1],
        [1, 1, 1, 1]],

       [[1, 1, 1, 1],
        [1, 1, 1, 1],
        [1, 1, 1, 1]]], dtype=int16)
>>> np.empty( (2,3) )
array([[ 0.,  0.,  0.],
       [ 0.,  0.,  0.]])
\end{python}

为了创建一个数字序列,numpy提供了一个和Python内建函数range相类似的函数,即numpy.arange,该方法返回一个一维数组。
\begin{python}
>>> np.arange( 10, 30, 5 )
array([10, 15, 20, 25])
>>> np.arange( 0, 2, 0.3 ) # it accepts float arguments
array([ 0. , 0.3, 0.6, 0.9, 1.2, 1.5, 1.8])
\end{python}

当给arange函数传入一个浮点数参数的时候,由于浮点数的精度是有限的,因此无法准确得知创建数组的元素个数。出于以上原因,numpy提供了一个更好用的函数numpy.linspace。linspace的用法和Matlab中的linspace用法基本类似,其两个参数的意义和arange一样,第三个参数表示元素的个数。它会根据元素的个数自动推导元素间的step。另外需要注意的是,arange函数的第二个参数不是闭区间,而linspace是闭区间。比如linspace(-5,5,1)最终产生的数组就包含了“5”这个点,而arange不会,需要注意其区别。
\begin{python}
>>> from numpy import pi
>>> np.arange( 0, 2, 0.3 ) # it accepts float arguments
array([ 0. , 0.3, 0.6, 0.9, 1.2, 1.5, 1.8])
>>> np.arange(10,30,5)
array([10, 15, 20, 25])
>>> from numpy import pi
>>> np.linspace(0,2,5)
array([ 0. ,  0.5,  1. ,  1.5,  2. ])
>>> np.arange(0,2,0.5)
array([ 0. ,  0.5,  1. ,  1.5])
>>> x=np.linspace(0, 2*pi, 10)
>>> x
array([ 0.        ,  0.6981317 ,  1.3962634 ,  2.0943951 ,  2.7925268 ,
        3.4906585 ,  4.1887902 ,  4.88692191,  5.58505361,  6.28318531])
\end{python}

另外还可以根据已知的数组的shape依样画葫芦创建一个新的数组。涉及到的函数有zeros\_like、empty\_like、ones\_like等。下面的示例中,a是shape为(3,2)的数组,通过ones\_like创建一个shape为(3,2),元素值全1数组b。
\begin{python}
>>> a=np.linspace(1. ,6. ,6).reshape(3,2)
>>> a
array([[ 1.,  2.],
       [ 3.,  4.],
       [ 5.,  6.]])
>>> b=np.ones_like(a)
>>> b
array([[ 1.,  1.],
       [ 1.,  1.],
       [ 1.,  1.]])
\end{python}

总结一下,如表\ref{fig:CreateArray}所示,numpy提供了大量的数组创建函数。创建数组的关键在于告知数组的形状,元素的类型,元素的值。元素的dtype可以不用显式给出,numpy可以自主推导合适的dtype。元素的值也可以不用暂时不用指定,用“占位”的方式来减少对动态增长数组的需求。

\begin{table}
\centering
\begin{tabular}{cp{10cm}}
\toprule
\tablehead{Function} & \tablehead{Instruction}\\
\midrule
array & 将输入数据(列表、元组、数组或者其他序列)转换为ndarray。推断出dtype或显式指定dtype。默认直接复制输入数据\\
asarray & 将输入转换为ndarray,若输入本身就是一个ndarray就不复制\\
arange & 类似内置的range,但返回的是一个ndarray而不是列表\\
ones、ones\_like & 根据指定的形状和dtype创建一个全1的数组。ones\_like以另一个数组为参数,并根据其形状和dtype创建一个全1数组\\
zeros、zeros\_like & 类似于ones和ones\_like,只不过产生的是全0数组而已\\
empty、empty\_like & 创建新数组,只分配内存空间但不填充任何值\\
eye、identity & 创建一个正方的$n\times n$单位矩阵(对角线为1,其余为0)\\
\bottomrule
\end{tabular}
\caption{数组创建函数}\label{fig:CreateArray}
\end{table}
\subsection{print数组}
当你print一个数组的时候,numpy会采用和打印Python嵌入式列表相类似的方式打印,为了方便查看,Numpy打印时添加了以下特征:
\begin{compactitem}
	\item 排在最后的axis打印方向是从左到右
	\item 排在倒数第二的axis打印方向是从上到下
	\item 其他axis打印方向是从上到下,但是会单独用一个空行来将不同的片区分开来,方便查看
\end{compactitem}

一维数组打印方式和列表一样,二维数组打印方式如矩阵形式,三维数组打印方式如矩阵的列表。

